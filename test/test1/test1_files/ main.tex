\documentclass{article}
\usepackage{tikz}
\usepackage{amsmath}
\usepackage{algorithm}
\usepackage{algpseudocode}
\usetikzlibrary{shapes.geometric, arrows}

\begin{document}

\title{Flowchart and Pseudocode for Python Calculator}
\author{}
\date{}
\maketitle

\section*{Flowchart}

\tikzstyle{startstop} = [rectangle, rounded corners, minimum width=3cm, minimum height=1cm,text centered, draw=black, fill=red!30]
\tikzstyle{io} = [trapezium, trapezium left angle=70, trapezium right angle=110, minimum width=3cm, minimum height=1cm, text centered, draw=black, fill=blue!30]
\tikzstyle{process} = [rectangle, minimum width=3cm, minimum height=1cm, text centered, draw=black, fill=orange!30]
\tikzstyle{decision} = [diamond, minimum width=3cm, minimum height=1cm, text centered, draw=black, fill=green!30]
\tikzstyle{arrow} = [thick,->,>=stealth]

\begin{tikzpicture}[node distance=2cm]

% Nodes
\node (start) [startstop] {Start};
\node (menu) [io, below of=start] {Display Menu};
\node (choice) [process, below of=menu] {Get User Choice};
\node (checkchoice) [decision, below of=choice, yshift=-1cm] {Choice = 5?};
\node (exit) [startstop, below of=checkchoice, yshift=-1cm] {Exit};
\node (operation) [process, right of=checkchoice, xshift=3cm] {Perform Operation};
\node (error) [io, below of=operation] {Display Error};
\node (result) [io, below of=error] {Display Result};
\node (loop) [startstop, below of=result] {Back to Menu};

% Arrows
\draw [arrow] (start) -- (menu);
\draw [arrow] (menu) -- (choice);
\draw [arrow] (choice) -- (checkchoice);
\draw [arrow] (checkchoice) -- node[anchor=east] {yes} (exit);
\draw [arrow] (checkchoice) -- node[anchor=south] {no} (operation);
\draw [arrow] (operation) -- (result);
\draw [arrow] (result) -- (loop);
\draw [arrow] (loop) -- (menu);
\draw [arrow] (operation) -- (error);
\draw [arrow] (error) -- (result);

\end{tikzpicture}

\newpage

\section*{Pseudocode}

\begin{algorithm}
\caption{Python Calculator Pseudocode}
\begin{algorithmic}[1]
\State \textbf{Function} add(x, y)
\State \hspace{1cm} \textbf{Return} x + y
\State \textbf{Function} subtract(x, y)
\State \hspace{1cm} \textbf{Return} x - y
\State \textbf{Function} multiply(x, y)
\State \hspace{1cm} \textbf{Return} x * y
\State \textbf{Function} divide(x, y)
\If{y = 0}
    \State \hspace{1cm} \textbf{Return} "Cannot divide by zero!"
\Else
    \State \hspace{1cm} \textbf{Return} x / y
\EndIf
\State
\State \textbf{Function} calculator()
\State \hspace{1cm} Print "Welcome to the Python Calculator!"
\While{True}
    \State \hspace{1cm} Print "Choose the operation you'd like to perform:"
    \State \hspace{1cm} Print "1. Add (+), 2. Subtract (-), 3. Multiply (*), 4. Divide (/), 5. Exit"
    \State \hspace{1cm} Input choice
    \If{choice = 5}
        \State \hspace{1cm} Print "Goodbye!"
        \State \hspace{1cm} Break
    \EndIf
    \If{choice \in [1, 2, 3, 4]}
        \State \hspace{1cm} Input num1, num2
        \If{choice = 1}
            \State \hspace{1cm} result = add(num1, num2)
        \ElsIf{choice = 2}
            \State \hspace{1cm} result = subtract(num1, num2)
        \ElsIf{choice = 3}
            \State \hspace{1cm} result = multiply(num1, num2)
        \ElsIf{choice = 4}
            \State \hspace{1cm} result = divide(num1, num2)
        \EndIf
        \State \hspace{1cm} Print "The result is: " + result
    \Else
        \State \hspace{1cm} Print "Invalid input, please select a valid option."
    \EndIf
\EndWhile
\end{algorithmic}
\end{algorithm}

\end{document}
